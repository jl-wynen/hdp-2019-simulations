\documentclass[a4paper, 14pt, fleqn, notitlepage]{scrartcl}

\usepackage[a4paper,
            vdivide={2cm,,2cm},
            hdivide={1cm,,1cm}
            ]{geometry}

\usepackage{xltxtra}
\usepackage[german]{babel}

\usepackage{amsmath}
\usepackage{amsfonts}
\usepackage{amssymb}
\usepackage{amsthm}
\usepackage{mathtools}
\usepackage{commath}            % for nicer differentials
\usepackage{bm}                 % bold math
\usepackage{dsfont}

\usepackage{datetime}
\usepackage{graphics}           % addition to above
\usepackage{float}              % place graphics with "H"
\usepackage{caption}
\usepackage{subcaption}
\usepackage{placeins}           % FloatBarrier
\usepackage[dvipsnames]{xcolor}
\usepackage{tcolorbox}


%---------------------------------------------------------
% general info

% ISO date format
\newdateformat{isodate}{\THEYEAR--\twodigit{\THEMONTH}--\twodigit{\THEDAY}}

\newcommand{\theauthor}{Jan-Lukas Wynen}
\newcommand{\thetitle}{Formelsammlung zur Merkursimulation}
\newcommand{\thedate}{\isodate\today}

\author{\theauthor}
\title{\vspace{-4em}\thetitle}
\date{\thedate}


%---------------------------------------------------------
% abbreviations
\newcommand{\unit}[1]{\,\text{#1}}
\newcommand{\unitr}{\frac{\vec{r}}{r}}


%---------------------------------------------------------
% miscellaneous

% custom emphasis
\newcommand{\empha}[1]{{\color{aiphiemph1}#1}}
\newcommand{\emphb}[1]{{\color{aiphiemph2}#1}}
\newcommand{\emphc}[1]{{\color{aiphiemph3}#1}}
\renewcommand{\emph}[1]{\empha{#1}}

% hide section numbers
\renewcommand\thesection{}

% don't put a space after , in math
\DeclareMathSymbol{,}{\mathord}{letters}{"3B}


% ---------------------------------------------------------
% document

\begin{document}
\sffamily
\pagestyle{empty}

\begin{center}
  \textbf{\textsf{\huge\thetitle}}
\end{center}

\begin{center}
  \color{black!50!white}{\rule{\textwidth}{1pt}}
\end{center}
\vspace{-2.5em}
\section{Physikalische Größen}

\begin{tabular}{ll}
  Gravitationskonstante & $G = 6,67 \cdot 10^{-11} \, \frac{\text{m}^3}{\text{kg}\,\text{s}^2}$\\
  Sonnenmasse & $M_\odot = 1,99 \cdot 10^{30} \unit{kg}$\\
  Lichtgeschwindigkeit & $c = 3,00 \cdot 10^{8} \, \frac{\text{m}}{\text{s}}$\\[1em]
  Schwarzschildradius d.\ Sonne & $\mathtt{r\_S} = r_S = \frac{2 G M_\odot}{c^2} = 2,95 \unit{km}$\\
  Spezifischer Drehimpuls & $\mathtt{rL2} = r_L^2 = \frac{\vec{L}^2}{m^2 c^2} = \frac{{(\vec{r} \times \vec{v})}^2}{c^2} = 8.19 \cdot 10^{13} \,\text{m}^2$\\[1em]
  \color{black!75!white}{Referenzlänge} & \color{black!75!white}{$R_0 = 10^{10} \unit{m}$}\\
  \color{black!75!white}{Referenzzeit} & \color{black!75!white}{$T_0 = 1 \unit{d}$}
\end{tabular}

\vspace{1em}
\begin{center}
  \color{black!50!white}{\rule{\textwidth}{1pt}}
\end{center}
\vspace{-1.5em}
\begin{minipage}[t]{1.0\paperwidth}
  \section{Gravitation}
\end{minipage}\\[0.6em]

\noindent
\begin{minipage}[t]{0.5\paperwidth}
  Klassische Gravitationskraft:
  \begin{align*}
    \vec{F} = - \frac{G m M_\odot}{r^2} \unitr = - \frac{mc^2}{2} \frac{r_S}{r^2} \unitr
  \end{align*}\\[0.5em]
  Allgemeine Relativitätstheorie (Näherung):
  \begin{align*}
    \vec{F} = - \frac{mc^2}{2} \frac{r_S}{r^2} \left(1 + a \frac{r_S}{r} + b \frac{r_L^2}{r^2}\right) \unitr
  \end{align*}
\end{minipage}
\begin{minipage}[t]{0.5\paperwidth}
  Reale werte:
  \begin{align*}
    a &= 0, \quad b = 3\\[0.5em]
    \frac{r_S}{r} &\approx 6 \cdot 10^{-8}, \quad \frac{r_L^2}{r^2} \approx 4 \cdot 10^{-8}
  \end{align*}
  In der Simulation:
  \begin{align*}
    a,\, b \gtrsim 10^{6}
  \end{align*}
\end{minipage}

\vspace{1em}
\begin{center}
  \color{black!50!white}{\rule{\textwidth}{1pt}}
\end{center}
\vspace{-1.5em}
\begin{minipage}[t]{0.5\paperwidth}
  \section{Bewegungsgleichung}

  Newtons Gleichung:
  \begin{align*}
    \vec{F} = m \vec{a}
  \end{align*}
  Ableitung (Änderung in der Zeit):
  \begin{align*}
    \vec{v}(t) &= \dot{\vec{r}}(t) \approx \frac{\vec{r}(t + \Delta t) - \vec{r}(t)}{\Delta t}\\[0.6em]
    \vec{a}(t) &= \dot{\vec{v}}(t) \approx \frac{\vec{v}(t + \Delta t) - \vec{v}(t)}{\Delta t}
  \end{align*}
\end{minipage}
%
\begin{minipage}[t]{0.5\paperwidth}
  \section{Euler-Verfahren}

  \begin{itemize}\setlength{\itemsep}{0.5em}
  \item[Haben:] $\vec{r}(t), \;\; \vec{v}(t)$
  \item[Wollen:] $\vec{r}(t + \Delta t), \;\; \vec{v}(t + \Delta t)$
  \item[1.] $\vec{a}(t) = \vec{F}(\vec{r}(t)) / m$
  \item[2.] $\vec{v}(t + \Delta t) = \vec{v}(t) + \vec{a}(t) \Delta t$
  \item[3.] $\vec{r}(t + \Delta t) = \vec{r}(t) + \vec{v}(t + \Delta t) \Delta t$
  \end{itemize}
\end{minipage}

\end{document}

%%% Local Variables:
%%% coding: utf-8
%%% mode: latex
%%% TeX-engine: xetex
%%% End: