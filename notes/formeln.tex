\documentclass[a4paper, 14pt, fleqn, notitlepage]{scrartcl}

\usepackage[a4paper,
            vdivide={2cm,,2cm},
            hdivide={1cm,,1cm}
            ]{geometry}

\usepackage{xltxtra}
\usepackage[german]{babel}

\usepackage{amsmath}
\usepackage{amsfonts}
\usepackage{amssymb}
\usepackage{amsthm}
\usepackage{mathtools}
\usepackage{commath}            % for nicer differentials
\usepackage{bm}                 % bold math
\usepackage{dsfont}

\usepackage{datetime}
\usepackage{graphics}           % addition to above
\usepackage{float}              % place graphics with "H"
\usepackage{caption}
\usepackage{subcaption}
\usepackage{placeins}           % FloatBarrier
\usepackage[dvipsnames]{xcolor}
\usepackage{tcolorbox}


%---------------------------------------------------------
% general info

% ISO date format
\newdateformat{isodate}{\THEYEAR--\twodigit{\THEMONTH}--\twodigit{\THEDAY}}

\newcommand{\theauthor}{Jan-Lukas Wynen}
\newcommand{\thetitle}{Formelsammlung zur Merkursimulation}
\newcommand{\thedate}{\isodate\today}

\author{\theauthor}
\title{\vspace{-4em}\thetitle}
\date{\thedate}


%---------------------------------------------------------
% abbreviations
\newcommand{\unit}[1]{\,\text{#1}}
\newcommand{\unitr}{\frac{\vec{r}}{r}}


%---------------------------------------------------------
% miscellaneous

% custom emphasis
\newcommand{\empha}[1]{{\color{aiphiemph1}#1}}
\newcommand{\emphb}[1]{{\color{aiphiemph2}#1}}
\newcommand{\emphc}[1]{{\color{aiphiemph3}#1}}
\renewcommand{\emph}[1]{\empha{#1}}

% hide section numbers
\renewcommand\thesection{}

% don't put a space after , in math
\DeclareMathSymbol{,}{\mathord}{letters}{"3B}


% ---------------------------------------------------------
% document

\begin{document}
\sffamily
\pagestyle{empty}

\begin{center}
  \textbf{\textsf{\huge\thetitle}}
\end{center}

\begin{center}
  \color{black!50!white}{\rule{\textwidth}{1pt}}
\end{center}
\vspace{-2.5em}
\section{Physikalische Gr\"o\ss en}

\begin{tabular}{ll}
  Gravitationskonstante & $G = 6,67 \cdot 10^{-11} \, \frac{\text{m}^3}{\text{kg}\,\text{s}^2}$\\
  & \\
  Sonnenmasse & $M_\odot = 1,99 \cdot 10^{30} \unit{kg}$\\
  & \\
  Lichtgeschwindigkeit & $c = 3,00 \cdot 10^{8} \, \frac{\text{m}}{\text{s}}$\\
  & \\
  K\"urzester Abstand Sonne-Merkur & $r_{\rm Merkur}=46\cdot 10^6$ km \\
    & \\
Merkur Geschwindigkeit dort & $v_{\rm Merkur}=59\, \frac{\text{km}}{\text{s}}$ \\
  & \\
  Schwarzschildradius d.\ Sonne & $r_S = \frac{2 G M_\odot}{c^2} = 2,95 \, \unit{km}$\\
  & \\
  Spezifischer Drehimpuls & $r_L^2 = \frac{\vec{L}_{\rm Merkur}^2}{m^2 c^2} = \frac{{(\vec{r}_{\rm Merkur} \times \vec{v}_{\rm Merkur})}^2}{c^2} = 8,19 \cdot 10^{13} \,\text{m}^2$\\
  & \\
  Kraftfaktor & $c_a = \frac{c^2r_S}{2}=1,33\cdot 10^{11} \ \frac{\text{km}^3}{\text{s}^2}$  \\
  & \\[1em]
  \color{black!75!white}{Referenzl\"ange} & \color{black!75!white}{$R_0 = 10^{10} \unit{m} = 10^{7} \unit{km}$}\\
  \color{black!75!white}{Referenzzeit} & \color{black!75!white}{$T_0 = 1 \unit{d}=8,84\cdot 10^4 \unit{s}$}
\end{tabular}

\vspace{1em}
\begin{center}
  \color{black!50!white}{\rule{\textwidth}{1pt}}
\end{center}
\vspace{-1.5em}
\begin{minipage}[t]{1.0\paperwidth}
  \section{Gravitation}
\end{minipage}\\[0.6em]

\noindent
\begin{minipage}[t]{0.6\paperwidth}
  Newton'sche Gravitationskraft:
  \begin{align*}
    \vec{F} = - \frac{G m M_\odot}{r^2} \unitr = - \frac{mc^2}{2} \frac{r_S}{r^2} \unitr= - m \frac{c_a}{r^2} \unitr
  \end{align*}\\[0.5em]
  
  \vspace{-1.3cm}
  
  $\Longrightarrow$ {\bf F\"uhrt zu geschlossenen Ellipsenbahnen}
    
  \vspace{0.5cm}
  
  Korrektur:
  \begin{align*}
    \vec{F} = - \frac{mc^2}{2} \frac{r_S}{r^2} \left(1 + a \frac{r_S}{r} + b \frac{r_L^2}{r^2}\right) \unitr
  \end{align*}
  
  mit
  
  \vspace{-1cm}
  
  $$
    \frac{r_S}{\bar r_{\rm Merkur}} \approx 6 \cdot 10^{-8}, \quad \frac{r_L^2}{r_{\rm Merkur}^2} \approx 4 \cdot 10^{-8}
$$
  
  
\end{minipage}
\begin{minipage}[t]{0.5\paperwidth}
  Allgemeine Relativit\"atstheorie:
  \begin{align*}
    a &= 0, \quad b = 3\\[1.em]
    \end{align*}
  In der Simulation:
  \begin{align*}
    a,\, b \gtrsim 10^{6}
  \end{align*}
\end{minipage}

\vspace{1em}
\begin{center}
  \color{black!50!white}{\rule{\textwidth}{1pt}}
\end{center}
\vspace{-1.5em}
\begin{minipage}[t]{\paperwidth}
{\Large Ziel: {\bf Berechnung der Bahnen des Merkurs f\"ur \\ {\phantom{Ziel:}}verschiedene Kraftgesetze}}

\vspace{0.5cm}

Wir brauchen: Zu jeder Zeit $t$ Ort $(\vec r)$ und Geschwindigkeit $(\vec v)$

\vspace{0.5cm}

also: $(\vec r(t),\vec v(t))\ \Longrightarrow \ (\vec r(t+\Delta t),\vec v(t+\Delta t))$

\vspace{0.5cm}


 {\bf Bewegungsgleichung:}  ${\vec{F} = m \vec{a}}$

\vspace{0.5cm}
wobei gilt
$$
    \vec{a}(t) = \frac{d\vec{v}(t)}{dt} \approx \frac{\vec{v}(t + \Delta t) - \vec{v}(t)}{\Delta t}\quad \mbox{und}\quad
    \vec{v}(t) = \frac{d\vec{r}(t)}{dt} \approx \frac{\vec{r}(t + \Delta t) - \vec{r}(t)}{\Delta t}
 $$

\vspace{0.5cm}

 Also (Eulerverfahren): 
 
\vspace{0.9cm}

 Kennt man die Kraft, so kennt man die Beschleunigung. 
  
\vspace{0.4cm}

\hspace{2cm}  $\vec{a}(t) = \vec{F}(\vec{r}(t)) / m = -c_a/r^2  \left(1 + a \frac{r_S}{r} + b \frac{r_L^2}{r^2}\right) \ (\vec r/r)$ 
  
\vspace{0.4cm}

\hspace{3.0cm} {\tt aMS = c\_a / vec\_rM\_old.mag**2 }

\hspace{3.0cm} {\tt aMS = aMS*( 1 + a*rS/vec\_rM\_old.mag + b*rL2/vec\_rm\_old.mag**2) }
  
\hspace{3.0cm} {\tt vec\_aMS = -  aMS * (vec\_rM\_old/vec\_rM\_old.mag)}
  
\vspace{0.9cm}

$\Longrightarrow$  \"Anderung der Geschwindigkeit 
  
\vspace{0.4cm}
 
\hspace{2cm}   $\vec{v}(t + \Delta t) = \vec{v}(t) + \vec{a}(t) \Delta t$
 
\vspace{0.4cm}

\hspace{3.0cm} {\tt vec\_vM\_new =   vec\_vM\_old +  vec\_ams * dt}

\vspace{0.9cm}

$ \Longrightarrow$  \"Anderung des Ortes 
   
\vspace{0.4cm}
 
\hspace{2cm}   $\vec{r}(t + \Delta t) = \vec{r}(t) + \vec{v}(t + \Delta t) \Delta t$
 
\vspace{0.4cm}

\hspace{3.0cm} {\tt vec\_rM\_new =   vec\_rM\_old +  vec\_vM\_new * dt}

 \end{minipage}
%

\vspace{2cm}


\end{document}

%%% Local Variables:
%%% coding: utf-8
%%% mode: latex
%%% TeX-engine: xetex
%%% End: